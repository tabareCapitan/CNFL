Households in the TVP program used 22.28\%, 40.35\%, and 37.37\% of their electricity in peak, mid-peak, and off-peak hours. Given this load profile, the average price per kW h under TVP is CRC 71.29, but it would have been CRC 92.36 under block pricing.\footnote{We do not have billing data. Instead, we calculated monthly billing based on our consumption data. Specifically, we calculated monthly billing data based on the pricing described in Table \ref{fig:table1}} In other words, the average price under TVP is much lower than the hypothetical average price that households in the TVP program would have paid had they been under block pricing.

We argue that our estimates must be interpreted as a local treatment effect that is “helpful to understand the early phases of these potential voluntary programs”. Since most TVP programs in the residential sector are voluntary, the scope of previous research on the topic is similar to ours. Our key contribution is to shed some light on the mechanism underlying the effect of TVP in total consumption. Specifically, we argue that the lower average price under TVP reduces total consumption both in our case and in previous work (a change along the demand curve). However, previous work has been conducted in settings in which technological changes in response to the TVP are much more likely to occur than in our case, which might lead to an inward shift of the demand curve. In the presence of a shift in the demand curve, households might consume less even if the new average price is lower.

Overall, this selection on gains creates large savings for those households that can take advantage of the TVP pricing (structural winners). In fact, households in the TVP program saved money (relative to counterfactual billing under block pricing) in about 93\% of the 315,265 monthly bills we calculated (see the distribution of the difference between billing under TVP and billing under block pricing in Figure \ref{fig:billing}. And these savings are economically significant. On average, each household saved CRC 148,836 per year relative to counterfactual billing under block pricing. Perhaps more striking, the aggregate savings for the households under TVP were CRC 3,910,537,980. These aggregate savings can also be interpreted as potential income lost by the electric utility. Our work shows the potential for significant losses if the benefits from shifting some consumption away from peak hours are not enough to offset the lost income from structural winners joining the TVP program.

\begin{figure}[ht]
  \caption{Distribution of the difference between billing under TVP and billing under block pricing}\label{fig:billing}
  \begin{center}
  {\includegraphics[width=1\textwidth]{./figures/billing.png}}
  \end{center}
\end{figure}

\FloatBarrier
